\documentclass[a4paper,12pt]{article}
\usepackage[utf8]{inputenc}
\usepackage[english]{babel}

% Paquets i altres coses incloses per Raúl
\usepackage[margin=2cm]{geometry}   % Grandaria i marges de la pàgina
\usepackage{hyperref}               % Per poder possar links
\usepackage{parskip}                % Per les indentacions
\usepackage{soul}                   % Per poder ratllar text
\setlength{\parindent}{0pt}			% Elimina l'indent
\setlength{\parskip}{0.2cm}			% Modifica l'espai entre paràgrafs

%opening
%\title{\textsc{Data Mining} \\ Definition and project assignment}

\begin{document}

\begin{center}
    \huge{\textsc{Data Mining} \\ Definition and project assignment}
\end{center}

%\maketitle

% \begin{abstract}
%
% \end{abstract}

\section{Name of group components}

    \begin{itemize}
    
        \item Ibáñez Pérez, Raúl
        \item Lao Tebar, Diego
        \item Lázaro Costa, Carles
        \item López Alcácer, Albert
        \item Ribes Marzá, Albert
        \item Roldán Montaner, Carlos

    \end{itemize}

\section{Data Source}

    We got our dataset from the Machine Learning Repository of the Center for Machine Learning And Intelligent Systems (Bren School of Information and Computer Science)\newline
    Repository url: \url{http://archive.ics.uci.edu/ml/datasets/Heart+Disease}\newline
    Dataset url: \url{http://archive.ics.uci.edu/ml/machine-learning-databases/heart-disease/} \newline

    We are going to use the Hungarian, Long Beach and Switzerland non processed .data sources.

\section{Process to get data (\textbf{TO BE DONE})}
    The process to get data was done by the doctors and analysts from Hungary, Long Beach and Switzerland hospitals.
    
    The data is a copy of the patient medical results donated by different medical centers. All the centers used the same method and representation to keep the data in order to maintain the data consistency.

\section{What data is about (\textbf{NEEDS REVISING})}

    Our data is a merge of four different datasets concerning heart disease diagnosis.
    Each instance collects different conditions or variables (numerical, binary and qualitative) of a patient
    which can be used to predict the presence of a heart disease in that patient.
    
    The locations the data was collected from are:

    \begin{itemize}

        \item Cleveland Clinic Foundation (The data is corrupted - Discarded)
        \item Hungarian Institute of Cardiology, Budapest
        \item V.A. Medical Center, Long Beach, CA
        \item University Hospital, Zurich, Switzerland

    \end{itemize}

    The principal investigator responsible for the data collection are:

    \begin{itemize}

        \item Andras Janosi, M.D.
        \item William Steinbrunn, M.D.
        \item Matthias Pfisterer, M.D.
        \item Robert Detrano, M.D., Ph.D.

    \end{itemize}

    The Cleveland data is corrupted, so we discarded it.

\section{Structure of data matrix (\textbf{TO BE COMPLETED})}

    \begin{itemize}

        \item \textbf{Number of records}:
        \begin{itemize}

            \item \st{Cleveland: 303} (Discarded)
            \item Hungarian: 294
            \item Switzerland: 123
            \item Long Beach VA: 200
            \item \textbf{Total: 617}

        \end{itemize}
        \item \textbf{Number of variables}: 76 (including the predicted one)
        \item \textbf{Number of numerical variables}:
        \item \textbf{Number of binary variables}:
        \item \textbf{Number of qualitative variables}:
        \item \textbf{Number and \% of missing data per each variable}:
        \item \textbf{\% of missing data in the whole data matrix}:

    \end{itemize}


\end{document}
