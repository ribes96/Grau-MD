\documentclass[a4paper,12pt]{article}
\usepackage[utf8]{inputenc}
\usepackage[english]{babel}

% Paquets i altres coses incloses per Raúl % % %
\usepackage{multicol}				% Per les columnes
\usepackage[margin=2cm]{geometry}   % Grandaria i marges de la pàgina
\usepackage{hyperref}               % Per poder possar links
\usepackage{parskip}                % Per les indentacions
\usepackage{soul}                   % Per poder ratllar text
\setlength{\parindent}{0pt}			% Elimina l'indent
\setlength{\parskip}{0.2cm}			% Modifica l'espai entre paràgrafs

%opening
%\title{\textsc{Data Mining} \\ Definition and project assignment}

\begin{document}

\begin{center}
    \huge{\textsc{Data Mining} \\ Definition and project assignment}
\end{center}

%\maketitle

% \begin{abstract}
%
% \end{abstract}

\section{Name of group components}

    \begin{itemize}
    
        \item Ibáñez Pérez, Raúl
        \item Lao Tebar, Diego
        \item Lázaro Costa, Carlos
        \item López Alcácer, Albert
        \item Ribes Marzá, Albert
        \item Roldán Montaner, Carlos

    \end{itemize}

\section{Data Source}

    We got our dataset from the Machine Learning Repository of the Center for Machine Learning And Intelligent Systems (Bren School of Information and Computer Science)\newline
    Repository url: \url{http://archive.ics.uci.edu/ml/datasets/Heart+Disease}\newline
    Dataset url: \url{http://archive.ics.uci.edu/ml/machine-learning-databases/heart-disease/} \newline

    We are going to use the Hungarian, Long Beach and Switzerland non processed .data sources.

\section{Process to get data}
    Data was downloaded from \url{http://archive.ics.uci.edu/ml/datasets/Heart+Disease}. 
    
    The process to get data was done by the doctors and analysts from Hungary, Long Beach and Switzerland hospitals.
    
    Each data record is a copy of the patient medical results. All the centers used the same unit representation for the medical analysis to keep the data consistency.

\section{What data is about}

    Our data is a merge of four different datasets concerning heart disease diagnosis.
    Each instance collects different conditions or variables (numerical, binary and qualitative) of a patient,
    which can be used to predict the presence of a heart disease in that patient.
    
    The data was collected from:

    \begin{itemize}

        \item Hungarian Institute of Cardiology, Budapest by Andras Janosi, M.D.
        \item University Hospital, Zurich, Switzerland by William Steinbrunn, M.D.
        \item University Hospital, Basel, Switzerland by Matthias Pfisterer, M.D.
        \item V.A. Medical Center, Long Beach and Cleveland Clinic Foundation (The data is corrupted - Discarded) by Robert Detrano, M.D., Ph.D.

    \end{itemize}

    The Cleveland data is corrupted, so we discarded it.

\section{Structure of data matrix}

    \begin{itemize}

        \item \textbf{Number of records}:
        \begin{itemize}

            \item \st{Cleveland: 303} (Discarded)
            \item Hungarian: 294
            \item Switzerland: 123
            \item Long Beach VA: 200
            \item \textbf{Total: 617}

        \end{itemize}
        \item \textbf{Number of variables}: 76 (including the predicted one)
        \item \textbf{Number of numerical variables}: 55
        \item \textbf{Number of binary variables}: 15
        \item \textbf{Number of qualitative variables}: 6
        \item \textbf{Number and \% of missing data per each variable}:
        
    \end{itemize}
    
    \begin{multicols}{2}
            \begin{tabular}{|l|c|c|} \hline

                \textit{ITEM}	& \textit{\#Entities}	& \textit{MissPercentage} \\
                \hline ID		&   0	&	0\% \\
                \hline CCF		&	0	&	0\% \\
                \hline AGE		&	0	&	0\% \\
                \hline SEX		&	0	&	0\% \\
                \hline PAINLOC	&	282	&	31,37\% \\
                \hline PAINEXER	&	282 & 	31,37 \% \\
                \hline RESTREL	&	286 &	31,81\% \\
                \hline PNCADEN	&	899 &	100\% \\
                \hline CP		&   0	&	0\% \\
                \hline TRESTBPS	&	59	&	06,56\% \\
                \hline HTN		& 	34	&	03,78\% \\
                \hline CHOL		& 	30	&	03,34\% \\
                \hline SMOKE	& 	669	&	74,42\% \\
                \hline CIGS		& 	420	&	46,72\% \\
                \hline YEARS	& 	432	&	48,05\% \\
                \hline FBS		& 	90	&	10,01\% \\
                \hline DM		& 	804	&	89,43\% \\
                \hline FAMHIST	&	422	& 	46,94\% \\
                \hline RESTECG	&	2	&	0,22\% \\
                \hline EKGMO	& 	53	&	05,90\% \\
                \hline EKGDAY	&	54	&	06,01\% \\
                \hline EKGYR	& 	53	&	05,90\% \\
                \hline DIG		& 	68	& 	07,56\% \\
                \hline PROP		& 	66	& 	07,34\% \\
                \hline NITR		& 	65	& 	07,23\% \\
                \hline PRO		&	63	& 	07,01\% \\ \hline
                
            
            \end{tabular}
            \begin{tabular}{|l|c|c|} \hline    
                
                \textit{ITEM} 	& \textit{\#Entities}	& \textit{MissPercentage} \\
                \hline DIURETIC	& 	82	&	09,12\% \\
                \hline PROTO	&	112		&	12,46\% \\
                \hline THALDUR	&	56		&	6,23\% \\
                \hline THALTIME	&	453		&	50,39\% \\
                \hline MET		& 	105		&	11,68\% \\
                \hline THALACH	&	55		&	6,12\% \\
                \hline THALREST	&	56		&	6,23\% \\
                \hline TPEAKBPS	&	63		&	7,01\% \\
                \hline TPEAKBPD	&	63		&	7,01\% \\
                \hline DUMMY	&	59		&	6,56\% \\
                \hline TRESTBPD	&	59		&	6,56\% \\
                \hline EXANG	&	55		&	6,12\% \\
                \hline XHYPO	&	58		&	6,45\% \\
                \hline OLDPEAK	&	62		&	6,90\% \\
                \hline SLOPE	&	308		&	34,26\% \\
                \hline RLDV5	&	425		&	47,27\% \\
                \hline RLDV5E	&	142		&	15,80\% \\
                \hline CA		&	608		&	67,63\% \\
                \hline RESTCKM	&	899		&	100\% \\
                \hline EXERCKM	&	898		&	99,89\% \\
                \hline RESTEF	&	871		&	96,89\% \\
                \hline RESTWM	&	869		&	96,66\% \\
                \hline EXEREF	&	897		&	99,78\% \\
                \hline EXERWM	&	894		&	99,44\% \\
                \hline THAL		&	477		&	53,06\% \\
                \hline THALSEV	&	769		&	85,54\% \\ \hline
                
            
            \end{tabular}
            
        \end{multicols}
        \begin{multicols}{2}
            
            \begin{tabular}{|l|c|c|} \hline

                \textit{ITEM} 	& \textit{\#Entities}	& \textit{MissPercentage} \\
                \hline THALPUL	&	855		&	95,11\% \\
                \hline EARLOBE	&	898		&	99,89\% \\
                \hline CMO		&	11	&	01,22\% \\
                \hline CDAY		&	9	&	01,00\% \\
                \hline CYR		&	9	&	01,00\% \\
                \hline NUM		&	0	&	0\% \\
                \hline LMT		&	275	&	30,59\% \\
                \hline LADPROX	&	236	&	26,25\% \\
                \hline LADDIST	&	246	&	27,36\% \\
                \hline DIAG		&	558	&	62,07\% \\
                \hline CXMAIN	&	235	&	26,14\% \\
                \hline RAMUS	&	567	&	63,07\% \\ \hline
                
            \end{tabular}
            
            \begin{tabular}{|l|c|c|} \hline

                \textit{ITEM} 	& \textit{\#Entities}	& \textit{MissPercentage} \\
                \hline OM1		&	271	&	30,14\% \\
                \hline OM2		&	572	&	63,63\% \\
                \hline RCAPROX	&	245	&	27,25\% \\
                \hline RCADIST	&	270	&	30,03\% \\
                \hline LVX1		&	19	&	02,11\% \\
                \hline LVX2		&	19	&	02,11\% \\
                \hline LVX3		&	19	&	02,11\% \\
                \hline LVX4		&	19	&	02,11\% \\
                \hline LVF		&	16	&	01,78\% \\
                \hline CATHEF	&	588	&	65,41\% \\
                \hline JUNK		&	780	&	86,76\% \\
                \hline NAME		&	0	&	0\% \\ \hline
                
            \end{tabular}
	\end{multicols}
    
    \begin{itemize}
        \item \textbf{\% of missing data in the whole data matrix}: 31,09\%

    \end{itemize}
    
\end{document}
